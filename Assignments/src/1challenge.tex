\documentclass[11pt]{article}

%%  Dimensions and URL
\usepackage[margin=1in]{geometry}
\usepackage{hyperref}

%%  Definitions
\renewcommand{\baselinestretch}{1.1}
\pagestyle{plain}


\begin{document}

\begin{center}
{\bfseries \LARGE Programming Challenge 1}
\end{center}

In python, the \texttt{random} module can generate pseudo random numbers.
For the purpose of this course, such numbers can be considered random.
In particular, \texttt{random.randrange(2)} produces random bits.
To use the \texttt{random} module, it is necessary to \texttt{import random}.
Using a loop, store \texttt{N} random bits in a \texttt{list} object.
\begin{verbatim}
import random

SampleSpaceSize = 10
NumberTrials = 10

TrialSequence = []
for TrialIndex in range(0, NumberTrials):
    TrialSequence.append(random.randrange(SampleSpaceSize))
\end{verbatim}
Then, look at the empirical distribution of the ratios of zeros and ones.
\begin{verbatim}
percent = []
for OutcomeIndex in range(0, SampleSpaceSize):
    percent.append(TrialSequence.count(OutcomeIndex) / float(NumberTrials))
print percent
\end{verbatim}
Explore how the empirical distribution changes as \texttt{N} increases 10.0, 100.0, 1000.0, 10000.0.

\end{document}

